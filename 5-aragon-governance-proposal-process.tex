% SPDX-License-Identifier: CC-BY-SA-4.0

\title{Aragon Governance Proposal Process}
% SPDX-License-Identifier: CC-BY-SA-4.0

\documentclass[
	10pt,
	a4paper,		
	numbers=noenddot,
	parskip=full
]{scrartcl}

%!TEX root = network-dao-charter.tex
% SPDX-License-Identifier: CC-BY-SA-4.0

% Additional Title Field
\usepackage{titling}
\makeatletter
\def\@subtitle{\@latex@warning@no@line{No Version Information Specified}}
\def\subtitle#1{\gdef\@subtitle{#1}}

% Page Geometry
\usepackage[	
	top=2.5cm,
	bottom=2.5cm,
	right=2.5cm,
	left=2.5cm
]{geometry}
\usepackage{booktabs}

% Coloring, Graphics, and Hyperlinks
\usepackage{xcolor,graphicx}
\definecolor{Aragon-Blue}{rgb}{0.2353,0.5412,1}
\usepackage[
	breaklinks=true,
	linktocpage=true,
	colorlinks=true, 
	menucolor={Aragon-Blue},
	linkcolor={black},
	citecolor={Aragon-Blue},
	urlcolor={Aragon-Blue}
]{hyperref}

% Git Version Information
\usepackage[local]{gitinfo2}

% Acronyms and Glossaries
\usepackage[
	acronyms,
	shortcuts,
	nopostdot,
	nogroupskip,
	nonumberlist,
	toc
]{glossaries-extra}
\setabbreviationstyle[acronym]{long-short}

\newglossary[klg]{key}{kld}{kdn}{Key Definitions}
\newglossary[llg]{add}{lld}{ldn}{Additional Definitions}
\makenoidxglossaries

% Custom Fonts
\usepackage{mathspec}
\setmainfont[
	Path			= ./fonts/Overpass/,
	Extension		= .ttf,
	UprightFont		= *-Regular,
	BoldFont		= *-Bold,
	ItalicFont		= *-Italic,
	BoldItalicFont	= *-BoldItalic
]{Overpass}
\setsansfont[
	Path			= ./fonts/Overpass/,
	Extension		= .ttf,
	UprightFont		= *-Regular,
	BoldFont		= *-Bold,
	ItalicFont		= *-Italic,
	BoldItalicFont	= *-BoldItalic
]{Overpass}
\setmathfont[
	Path			= ./fonts/RobotoMono/,
	Extension		= .ttf,
	UprightFont		= *-Regular,
	BoldFont		= *-Bold,
	ItalicFont		= *-Italic,
	BoldItalicFont	= *-BoldItalic
]{RobotoMono}
\setmonofont[
	Path			= ./fonts/RobotoMono/,
	Extension		= .ttf,
	UprightFont		= *-Regular,
	BoldFont		= *-Bold,
	ItalicFont		= *-Italic,
	BoldItalicFont	= *-BoldItalic
]{RobotoMono}

% Deeper Enumerations
\usepackage{enumitem}
\setlist{
	nosep,
	topsep=4pt,
	parsep=0.5\parskip
}
\setlistdepth{5}
\renewlist{enumerate}{enumerate}{5}
\setlist[enumerate,1]{label=\arabic*.}
\setlist[enumerate,2]{label=\alph*.}
\setlist[enumerate,3]{label=\roman*.}
\setlist[enumerate,4]{label=\arabic*.}
\setlist[enumerate,5]{label=\alph*.}

%!TEX root = network-dao-charter.tex
% SPDX-License-Identifier: CC-BY-SA-4.0

% General
\newcommand{\antTokenAddr}{\ethaddress{0xa117000000f279d81a1d3cc75430faa017fa5a2e}}
\newcommand{\aragonCourtAddr}{\ethaddress{0xfb072baa713b01ce944a0515c3e1e98170977daf}}

% Aragon Network Main DAO
\newcommand{\mainDaoAddr}{\ethaddress{0x9c1d24318966793a68e6005eb6b27edace3f28b8}}
\newcommand{\mainDaoTokenAddr}{\antTokenAddr}

% Executive Sub-DAO
\newcommand{\execSubDaoAddr}{\ethaddress{0xe0caf0349cff049f610c3f04f5d7889f58d98d8c}}
\newcommand{\execSubDaoTokenAddr}{\ethaddress{0x613a126c20632c99afd01b044fe13e97b76eeb5a}}
% Election: https://voice.aragon.org/processes/#/0x87cd66fd8d890849a7480a5460d5d2902bed9e30ff1e229c7c548b0cddae58cf
\newcommand{\execSubDaoMemberAddrI}{\ethaddress{0x8B580433568E521ad351b92b98150c0C65ce69B7}}
\newcommand{\execSubDaoMemberAddrII}{\ethaddress{0x3E31155a1c17c9F85e74828447aec412090a4622}}
\newcommand{\execSubDaoMemberAddrIII}{\ethaddress{0x1E8eE48D0621289297693fC98914DA2EfDcE1477}}

% Tech Sub-DAO
\newcommand{\techSubDaoAddr}{-}
\newcommand{\techSubDaoTokenAddr}{0x731B540B83292734F866fF1850532DF1D7A1F80e}
% Election: https://voice.aragon.org/processes/#/0xfc50351d7c1bbd0ab5f4ad17ee839a41a49e7fb4a5ebd47fde2c51b208372193
\newcommand{\techSubDaoMemberAddrI}{\ethaddress{0x806A940E1C431aa36077c4f6fB3e7d36CEF2a9A7}}
\newcommand{\techSubDaoMemberAddrII}{\ethaddress{0xDd9d481BE2ff4957f8Db967F8b301d23bCd4aE64}}
\newcommand{\techSubDaoMemberAddrIII}{\ethaddress{0xCE3d8791c1bdaCc6b8e1a52B4E6aC140F8a2C8c3}}

% Compliance Sub-DAO
\newcommand{\cmplSubDaoAddr}{\ethaddress{0xbe39e9cb1daa8ee8838d6a93d360f7ea7b8373c2}}
\newcommand{\cmplSubDaoTokenAddr}{\ethaddress{0x8aa971084ed42fc3452d34c5aec4878c28dd7cd0}}
% Election: https://voice.aragon.org/processes/#/0xea559447e36ba9ee53e2b0bd8441e2cc4f6f50b69f98cf8702324fe6ca2cb82b)
\newcommand{\cmplSubDaoMemberAddrI}{\ethaddress{0x285fdc3cbf953715ab925ca82ae1e2087de6509a}}
\newcommand{\cmplSubDaoMemberAddrII}{\ethaddress{0x9c05Ea2B1f6BcAfF36Eae0574123D4CC092C6B5C}}
\newcommand{\cmplSubDaoMemberAddrIII}{\ethaddress{0x75cD58A01883C6A115A1293C4Dd4CE67D11928f0}}

\subtitle{\gitDescribe}
\author{Aragon Network}
\date{\today}

%!TEX root = network-dao-charter.tex
% SPDX-License-Identifier: CC-BY-SA-4.0

% Acronyms and definitions have to be defined here so that the can be used in the text later.

%%%%%%%%%%
% Acronyms
%%%%%%%%%%
\newacronym{AGP}{AGP}{Aragon Governance Proposal}
\newacronym{AN}{AN}{\glslink{AragonNetwork}{Aragon Network}}\glsunset{AN}
\newacronym{ANT}{ANT}{\glslink{AragonNetwork}{Aragon Network} \glslink{AragonNetworkToken}{Token}}\glsunset{ANT}
\newacronym{DAO}{DAO}{\glslink{DecentralizedAutonomousOrganization}{Decentralized Autonomous Organization}}\glsunset{DAO}
\newacronym{ENS}{ENS}{\glslink{EthereumNameService}{Ethereum Name Service}}\glsunset{ENS}
\newacronym{IPFS}{IPFS}{\glslink{IPFSNetwork}{InterPlanetary File System}}

%%%%%%%%%%%%%%%%%
% Key definitions
%%%%%%%%%%%%%%%%%
\newglossaryentry{ActiveDAOs}{
	type=key,
	name={Active Aragon \acp{DAO}},
	description={
		\acp{DAO} created using the smart contract provided by the Aragon Network, that have enacted at least 1 decision in the past month.
	}
}

\newglossaryentry{AragonNetwork}{
	type=key,
	name={\acl{AN}},
	description={
		a network of \glspl{ANTHolder} and the organisations and technology created by them to pursue the goals and live the values defined in the \href{https://github.com/aragon/AGPs/blob/master/AGPs/AGP-0.md}{Aragon Manifesto}. 
		Including but not limited to: The Aragon Network \ac{DAO}, \glspl{ANTHolder}, their affiliates, employees, contributors, licensors and service providers, and their respective officers, directors, employees, contractors, agents, licensors, suppliers, parent companies, successors, subsidiaries, affiliates, agents, representatives.
	}
}

\newglossaryentry{AragonNetworkCharter}{
	type=key,
	name=\acl{AN} Charter,
	description={the whole of this document (``this Charter''), which consists of the human-interpreted rules of the \gls{AragonNetwork} and is subdivided in the following human-readable documents:
		\begin{itemize}[noitemsep]
			\item \hyperref[chap:ANDAOAgreement]{\textbf{The Aragon Network DAO Agreement}}: the responsibilities and rights of \glspl{ANTHolder}.
			\item \hyperref[chap:AragonManifesto]{\textbf{The Aragon Manifesto}}: the values and the mission of the \gls{AragonNetwork}.
			\item \hyperref[chap:CommunityGuidelines]{\textbf{The Community Guidelines}}: the norms of behaviour between \glspl{ANTHolder}.
			\item \hyperref[chap:AGPProcess]{\textbf{The \ac{AGP} Process}}: the process used by \glspl{ANTHolder} to govern the \gls{AragonNetworkDAO}
		\end{itemize}
	}
}		

\newglossaryentry{AragonNetworkDAO}{
	type=key,
	name={\acl{AN} \ac{DAO}},
	description={
		a \ac{DAO} (an open coordination protocol based on open blockchain technology) that aims to facilitate the governance of the Aragon Network’s tools, infrastructure and other common resources. The Aragon Network \ac{DAO} is governed by \glspl{ANTHolder} according to the \ac{AGP} Process and includes one or more Sub-\acp{DAO} also binded by this Charter.
	}
}

\newglossaryentry{AragonNetworkToken}{
	type=key,
	name={\ac{ANT}},
	description={
		the native token of the \gls{AragonNetwork} - a transferable ERC-20 token with the contract address: \ethaddress{0xa117000000f279d81a1d3cc75430faa017fa5a2e}.
		ANT is used for the governance of the Aragon Network DAO and other forms of participation in the Aragon Network.
	}
}

\newglossaryentry{ANTHolder}{
	type=key,
	name={\ac{ANT} Holder},
	description={
		any entity that owns or controls an Ethereum account that controls a positive balance of \ac{ANT}.
	}
}

\newglossaryentry{AragonCourt}{
	type=key,
	name={Aragon Court},
	description={
		a smart contract-based dispute resolution system governed by \glspl{ANTHolder}.
	}
}

\newglossaryentry{AragonVoice}{
	type=key,
	name={Aragon Voice},
	description={A token-based voting client with universally verifiable results and deterministic execution.
	}
}

\newglossaryentry{CommunityMember}{
	type=key,
	name={Community Member},
	description={
		any \ac{ANT} Holder participating in the communication platforms, including Aragon’s Discord server, Aragon’s Telegram group, Aragon Forum, Aragon Network \ac{DAO}, online and offline gatherings, and/or other groups and platforms of the Aragon Network.
	}
}

\newglossaryentry{FinancialActions}{
	type=key, 
	name={Financial Actions},
	description={
		any on-chain action that transfers, exchanges, or stakes tokens held by the \ac{AN} \ac{DAO} or Sub-\acp{DAO}.
	}
}


%%%%%%%%%%%%%%%%%%%%%%%%
% Additional Definitions
%%%%%%%%%%%%%%%%%%%%%%%%

\newglossaryentry{AragonNetworkCommunityPlatforms}{
	type=add,
	name={Aragon Network Community Platforms},
	description={
		the platforms used for regular communication by the \glspl{ANTHolder} and managed by either the Aragon Association or the \ac{AN} \ac{DAO}. They include but are not restricted to Aragon’s Discord Server, Aragon’s Forum, Aragon’s GitHub repository, and online and offline events organised under the Aragon brand.
	}
}

\newglossaryentry{AragonNetworkOrganization}{
	type=add,
	name={Aragon Network organization},
	description={
		the Aragon organization addressed by the \ac{ENS} name\\ \ethaddress{network.aragonid.eth}, encompassing all apps installed on this organization.
	}
}

\newglossaryentry{DecentralizedAutonomousOrganization}{
	type=add,
	name=\glslink{DAO}{DAO},
	description={
		an open-source blockchain protocol governed by a set of rules, with fluid affiliation or membership through participation, and that automatically execute certain actions without the need for intermediaries.
	}
}

\newglossaryentry{EthereumNameService}{
	type=add,
	name={\ac{ENS}},
	description={
		a smart contract-based naming system. \ac{ENS} names are registered using the registry deployed at the Ethereum address \ethaddress{0x00000000000C2E074eC69A0dFb2997BA6C7d2e1e}.
	}
}

\newglossaryentry{Escrow}{
	type=add,
	name={Escrow},
	description={
		a contractual arrangement in which a third party (the stakeholder or escrow agent) receives and disburses money or property for the primary transacting parties, with the disbursement dependent on conditions agreed to by the transacting parties.
	}
}

\newglossaryentry{EthereumBlockchain}{
	type=add,
	name={Ethereum blockchain},
	description={
		The blockchain referred to as ``Ethereum'' by the owner of the Ethereum trademark, the Ethereum Foundation (Stiftung Ethereum).
	}
}

\newglossaryentry{Guardian}{
	type=add,
	name={Guardian},
	description={
		natural person (represented by a wallet address) that is eligible to be drafted to rule upon disputes in \gls{AragonCourt}. 
		If a guardian is summoned, it must vote to allow or block the action being disputed, and can be rewarded (earn tokens) or punished (lose tokens) depending if they voted with the majority (default interpretation) or against it. 
		The protocol is designed so that Guardians do not know each other and do not have contact when they are summoned.
	}
}


\newglossaryentry{IPFSNetwork}{
	type=add,
	name={\ac{IPFS} network},
	description={
		A peer-to-peer hypermedia protocol, whose project website is located at \href{https://ipfs.io}{https://ipfs.io}. 
		The public network makes files accessible to any internet-connected device running a compatible implementation of the \ac{IPFS} software.
	}
}

\newglossaryentry{Metagovernance}{
	type=add,
	name={Metagovernance (``Metagov'')},
	description={
		refers to the process(es) to change the governance processes of the \gls{AragonNetworkDAO}.
	}
}

\newglossaryentry{NetworkContract}{
	type=add,
	name={Network contract},
	description={
		The computer-interpreted source code of a smart contract governed by \glspl{ANTHolder}.
	}
}

\newglossaryentry{NetworkParameters}{
	type=add,
	name={Network parameters},
	description={
		The current parameters of smart contracts governed by \glspl{ANTHolder}, which can be modified without amending the source code of the smart contract itself.
	}
}
 
\newglossaryentry{Proposal}{
	type=add,
	name={Proposal},
	description={
		any suggested amendment to the Network contracts, Network parameters, the Agreements of the \gls{AragonNetwork} (i.e. \gls{AragonNetworkCharter} and composing agreements), and Agreements of any Sub-\acp{DAO}.
	}
}

\newglossaryentry{SmartContract}{
	type=add,
	name={Smart contract},
	description={
		a software program deployed to a blockchain.
	}
}

\newglossaryentry{OffChain}{
	type=add,
	name={Off-Chain},
	description={
		refers to any action or data entry that is not recorded on a blockchain or other type of distributed ledger where the smart contract (or version of it) is deployed or connected to so as to have verifiable and tamper-proof records.
	}
}

\newglossaryentry{OnChain}{
	type=add,
	name={On-Chain},
	description={
		refers to any action or data entry that is recorded on a blockchain or other type of distributed ledger where the smart contract (or version of it) is deployed or connected to so as to have verifiable and tamper-proof records.
	}
}

\newglossaryentry{OptimisticGovernance}{
	type=add,
	name={Optimistic Governance},
	description={
		the process of decisions being enacted with a time delay which allows time for a decision to be challenged. If a decision is not disputed then the change is enacted at the end of the time delay. This allows for much faster decision making and strategy decisions to be made.
	}
}





\begin{document}
\mytitle

\section{Purpose}
The purpose of the \acf{AGP} process is to provide a structured process for making changes to the shared resources of the \gls{AragonNetwork}.
For these shared resources, governance processes are needed to grant or deny access and approve or reject proposed changes.

\section{Governance Bodies}

The following bodies will form part of the \ac{AGP} process and are collectively referred to as the \gls{AragonNetworkDAO}, composed of a Main \ac{DAO} and Sub-\acp{DAO}.

\subsection{Main \ac{DAO}}

The Main \ac{DAO} is an Aragon Govern \ac{DAO} that uses \gls{AragonVoice} voting to validate the community’s sentiment and then scheduling for on-chain execution on Govern, enabling \glspl{ANTHolder} to both exercise direct democracy (directly propose and approve proposals of any kind) and elect members to existing or new Sub-\acp{DAO}.


\subsection{Sub-\acp{DAO}} 

The following Sub-\acp{DAO} exist and are controlled by the Main \ac{DAO}.


\subsubsection*{Executive Sub-\ac{DAO}}

The Executive Sub-\ac{DAO} is an Aragon Govern \ac{DAO}, with permission for treasury management of the Operations Vault.
\begin{enumerate}
	\item The Executive Committee \ac{DAO} is governed on a do-ocracy basis (Optimistic Governance) by the Executive Committee Members.
	\item Members of the Executive \ac{DAO} (``Executive Committee Members'') are elected by \glspl{ANTHolder} through a vote on the Main \ac{DAO}.
\end{enumerate}


\subsubsection*{Compliance Sub-\ac{DAO}}

The Compliance Sub-\ac{DAO} is an Aragon Govern \ac{DAO}, with permission to veto proposals and actions in the Main \ac{DAO} and Sub-\acp{DAO} that represent a breach of this Charter and/or directly harm the \gls{AragonNetwork}.
\begin{enumerate}
	\item The above shall not include actions or proposals to add and/or remove wallet addresses from the Compliance \ac{DAO}.
	\item The compliance \ac{DAO} is governed on a do-ocracy basis (Optimistic Governance) by the Compliance Committee Members.
	\item Members of the Compliance \ac{DAO} (``Compliance Committee Members'' i.e. wallets with permission to veto proposals through the Compliance \ac{DAO}) are elected by \glspl{ANTHolder} through a vote on the Main \ac{DAO}.
\end{enumerate}


\subsubsection*{Tech Committee}

The Tech Committee is an off-chain committee charged with deploying accepted proposals that contain new code (and managing the access to the necessary Web2 platforms to perform this task).
The Tech Committee also has the authority to remove proposals that do not meet technical quality standards as well as to determine which proposals should be subject to a 3rd party independent technology audit.
\begin{enumerate}
	\item The Tech Committee decides on the need (or lack thereof) for audits on proposals using \gls{AragonVoice} and a majority vote.
	\item Members of the Tech \ac{DAO} (``Tech Committee Members'') are elected by \glspl{ANTHolder} through a vote on the Main \ac{DAO}.
\end{enumerate}

\subsection{Aragon Court}
Used to challenge the actions scheduled on both the Main \ac{DAO} and any Sub-\ac{DAO}, and directly from the Voice UI / or the custom UI for AN \ac{DAO}.
	


\section{Separation of Powers}

\begin{enumerate}
	\item The same members shall not be part of more than one of the following: the Executive Sub-\ac{DAO}, Compliance Sub-\ac{DAO}, Tech Committee, and \glspl{Guardian} in \gls{AragonCourt} mainnet.
	\item When a decentralised identity solution is integrated into the \ac{DAO}, members of these committees shall prove their unique identity using said solution.
\end{enumerate}


\section{Formation of new Sub-DAOs}

\begin{enumerate}
	\item The Main \ac{DAO} can deploy a new Sub-\ac{DAO}, be given control over an already deployed \ac{DAO}, dissolve an existing Sub-\ac{DAO}, or spin-off a Sub-\ac{DAO} as an independent \ac{DAO} at any point in time.
	\item Each Sub-\ac{DAO} shall have its own operating agreement outlining at a minimum the responsibilities of its members which shall be listed as an Appendix to the \gls{AragonNetworkCharter}.
	\item In case of conflict between agreements, The \gls{AragonNetworkCharter} shall prevail over any Sub-\ac{DAO} operating agreement.
\end{enumerate}


\section{Treasuries \& Permissions}

\subsubsection*{Investment Vault} The reserve treasury of the \gls{AragonNetwork}.

Used to fund Sub-\acp{DAO} and/or to obtain yield.
\begin{enumerate}
	\item Any \ac{ANT} Holder can make proposals for Financial Proposals as described in the Financial Proposals section of this \ac{AGP} process document.
\end{enumerate}


\subsubsection*{Operations Vault} The main treasury of the \gls{AragonNetwork}.

Used to fund operations and make strategic fundings.
\begin{enumerate}
	\item Any \ac{ANT} Holder can make proposals for Financial Proposals as
	described in the Financial Proposals section of this \ac{AGP} process
	document.
	\item And any wallet with permissions for the Executive \ac{DAO} (i.e. any member of the Executive \ac{DAO}) can program and delete Financial Actions in the Executive Sub-\ac{DAO} as per the Executive Sub-\ac{DAO} operating agreement.
\end{enumerate}

\subsubsection*{Additional Treasuries}
\begin{enumerate}
	\item Sub-\acp{DAO} are not allowed to create additional treasuries under the exclusive control of the Sub-\ac{DAO} (and outside of the control of the Main \ac{DAO}) unless approved by a majority vote of \glspl{ANTHolder}.
	\item The Main \ac{DAO} may create additional treasuries through a majority vote of \glspl{ANTHolder}.
\end{enumerate}  


\section{Proposals}

\begin{enumerate}
	\item Any \ac{ANTHolder} can create a Proposal in the Main \ac{DAO}.
	\item Any Sub-\ac{DAO} may remove a scheduled proposal at any time (e.g. Veto right) should they have the power to do so as per their operating agreement and as per this Charter.
	\item All proposals must comply with the Requirements for Proposals \& the specific format and process for the type of proposal as follows:
\end{enumerate}

\subsection{The Requirements for Proposals}

\begin{enumerate}
	\item \textbf{Public deliberation:} all proposals must be shared during the public deliberation phase in the Aragon Forum and linked in the Aragon Discord Server, unless these services are unavailable.
	\item \textbf{Voting period:} the vote on \gls{AragonVoice} must be at least 7 days.
	\item \textbf{Scheduling:} once approved, proposals are automatically scheduled for execution 5 days after the end of the vote.
	\item \textbf{Collateral:} the proposer must put 50 \ac{ANT} as collateral during the voting period.
	This collateral might be slashed if the proposal is challenged in \gls{AragonCourt} and ruled to violate any provision of this Charter.
\end{enumerate}


\subsection{Types of Proposals}

\begin{enumerate}		
	\item \textbf{Financial Proposals:} any type of financial transfer of any asset that has a monetary value from the Main \ac{DAO} or any Sub-\ac{DAO}.
	\item \textbf{Election Proposals:} any type of proposal that gives/removes wallet permissions from the Sub-\acp{DAO}
	\item \textbf{Other Proposals:} any other type of proposal (including Metagovernance Proposals, code submissions, parameter changes, etc).
\end{enumerate}

\subsection{Process for Financial Proposals and/or Other Proposals} 
Proposals must pass through the following phases in sequential order:


\subsubsection*{Public deliberation}
\begin{enumerate}
\item A post with the draft of the proposal is posted in the Aragon forum for a minimum of 7 days and maximum of 14 days with the format:
	\begin{enumerate}
		\item Title of Proposal (in the format ``Financial Proposal:[title]'')
		\item Description of the Action
		\item Description of why the author believes it will help to increase the number of Active Aragon \acp{DAO}
		
		\begin{enumerate}
			\item By default, all Financial Proposals involving a deliverable should use an Escrow (see Additional Definitions).
			In cases where the use of an Escrow is omitted, a justification must be included in the Description.
			\item Greet.me shall be the default Escrow provider for Financial Proposals.
		\end{enumerate}
	
		\item And suggested optional: ETH Wallet address of the author(s) and/or other identifiers
			
	\end{enumerate}
\end{enumerate}

\subsubsection*{Voting}
\begin{enumerate}
\item The proposal (or a revised version of the proposal incorporating the community’s feedback) is posted for a vote on \gls{AragonVoice}.
	\begin{enumerate}
		\item For calculating voting power, 1 \ac{ANT} token = 1 Vote.
	\end{enumerate}
\end{enumerate}

\subsubsection*{Approval} 
\begin{enumerate}
	\item a proposal is deemed approved and scheduled for execution if the following conditions are reached:
	\begin{enumerate}
		\item Quorum: a minimum of five thousand (5,000) \ac{ANT} has been used to vote.
		\item  Support: The vote shall be deemed as “passed” with a simple majority (>50\%) of the participating \ac{ANT} having voted in favour.
		\item When a proposal includes a code submission, the Tech Committee has up to 14 days to decide and communicate via a post in Aragon Forum whether a proposal will be:
		\begin{enumerate}
			\item Accepted and incorporated.
			\item Submitted to a 3rd party audit to determine its safety (conditional on the Operations Vault having the necessary funds).
			\item Rejected as malicious, technically infeasible, or economically infeasible (if an audit is required and the Operations Vault lacks the necessary funds to cover the costs of the audit).
		\end{enumerate}
	\end{enumerate}
\end{enumerate}


\subsection{Process for Elections}

Elections must pass through the following phases in sequential order.

\subsubsection*{Public deliberation phase} 

\begin{enumerate}
	\item A post with the draft of the proposal is posted in the Aragon forum for a minimum of 10 days and maximum of 30 days with the format:
	\begin{enumerate}
		\item \textit{Title of Proposal (in the format “Election:[title]”)}
		\item \textit{Description of the Action (including what permissions will be given/removed for which \ac{DAO}/Sub-\ac{DAO})}
		\item \textit{In the event of the election being proposed before the stipulated in the agreements of the Sub-\ac{DAO}, add a description of why an advanced election is needed}
		\item \textit{And suggested optional: ETH Wallet address of the author(s) and/or other identifiers.}
	\end{enumerate}
\end{enumerate}

\subsubsection*{Sourcing Candidates}

\begin{enumerate}
	\item Candidates can be proposed (and/or propose themselves) by replying to the Forum post (only one candidate per post reply) in the following format:
	\begin{enumerate}
		\item \textit{Identifier: their ETH wallet address (and optional their name, discord handle, twitter handle, and other identifiers)}
		\item \textit{Rationale: description of why they are an ideal candidate for the position}
	\end{enumerate}
\end{enumerate}


\subsubsection*{Voting}
\begin{enumerate}
	\item The top 10 candidates with the highest number of upvotes in the Aragon Forum will be put forward to a vote using \gls{AragonVoice}.
	\begin{enumerate}
		\item In the event of the election being proposed before the frequency stipulated in the operating agreement of the Sub-\ac{DAO} (if any), the vote must also include the option to keep the current permissions even if the current wallets were not amongst the top 10 candidates.
	\end{enumerate}
\end{enumerate}

\subsubsection*{Approval \& Execution}
\begin{enumerate}
	\item In the exceptional case that two proposed users gain the same number of votes, the winner will be the candidate who reached the tieing number of votes first.
	\begin{enumerate}
		\item E.g. Candidate A and B both tied at 7 votes, Candidate A wins because it reached 7 votes a day before Candidate B.
		If votes are submitted in the same block, repeat the vote.
	\end{enumerate}

\end{enumerate}


\section{Disputes}

\begin{enumerate}
	\item Disputes between members that can not be addressed through facilitation or mediation, and disputes related to proposals shall be resolved using \gls{AragonCourt}.
	\item The losing party shall reimburse the winning party for any \gls{AragonCourt} fees incurred by the winning party. Failing that, the Executive \ac{DAO} shall reimburse said fees.
\end{enumerate}

\end{document}