%!TEX root = network-dao-charter.tex
% SPDX-License-Identifier: CC-BY-SA-4.0

\charterchapter{Aragon Governance Proposal Process}
\label{chap:AGPProcess}

The purpose of the \acf{AGP} process is to provide a structured process for making changes to the shared resources of the \gls{AragonNetwork}.
For these shared resources, governance processes are needed to grant or deny access and approve or reject proposed changes.

\begin{enumerate}
	
	\sectionitem{Governance Bodies}
	\begin{enumerate}
		
		\item The following bodies will form part of the \ac{AGP} process and are collectively referred to as the \gls{AragonNetworkDAO}, composed of a Main \ac{DAO} and Sub-\acp{DAO}.
		\item \textbf{Main \ac{DAO}:} an Aragon Govern \ac{DAO} that uses \gls{AragonVoice} voting to validate the community’s sentiment and then scheduling for on-chain execution on Govern, enabling \glspl{ANTHolder} to both exercise direct democracy (directly propose and approve proposals of any kind) and elect members to existing or new Sub-\acp{DAO}.
		\item \textbf{Sub-\acp{DAO}:} \acp{DAO} controlled by the Main \ac{DAO}.
		At launch, the following Sub-\acp{DAO} will exist:
		\begin{enumerate}
			
			\item \textbf{Executive Sub-\ac{DAO}:} an Aragon Govern \ac{DAO}, with permission for treasury management of the Operations Vault.
			\begin{enumerate}
				\item The Executive Committee \ac{DAO} is governed on a do-ocracy basis (Optimistic Governance) by the Tech Committee Members.
				\item Members of the Executive \ac{DAO} (``Executive Committee Members'') are elected by \glspl{ANTHolder} through a vote on the Main \ac{DAO}.
			\end{enumerate}
		
			\item \textbf{Compliance Sub-\ac{DAO}:} an Aragon Govern \ac{DAO}, with permission to veto proposals and actions in the Main \ac{DAO} and Sub-\acp{DAO} that represent a breach of this Charter and/or directly harm the \gls{AragonNetwork}.
			\begin{enumerate}
				\item The above shall not include actions or proposals to add and/or remove wallet addresses from the Compliance \ac{DAO}.
				\item The compliance \ac{DAO} is governed on a do-ocracy basis (Optimistic Governance) by the Compliance Committee Members.
				\item Members of the Compliance \ac{DAO} (``Compliance Committee Members'' i.e. wallets with permission to veto proposals through the Compliance \ac{DAO}) are elected by \glspl{ANTHolder} through a vote on the Main \ac{DAO}.
			\end{enumerate}
			
			\item \textbf{Tech Committee:} an off-chain committee charged with deploying accepted proposals that contain new code (and managing the access to the necessary Web2 platforms to perform this task).
			The Tech Committee also has the authority to remove proposals that do not meet technical quality standards as well as to determine which proposals should be subject to a 3rd party independent technology audit.
			\begin{enumerate}
				\item The Tech Committee decides on the need (or lack thereof) for audits on proposals using \gls{AragonVoice} and a majority vote.
				\item Members of the Tech \ac{DAO} (``Tech Committee Members'') are elected by \glspl{ANTHolder} through a vote on the Main \ac{DAO}.
			\end{enumerate}
		
		\end{enumerate}
		
		\item \textbf{Aragon Court:} Used to challenge the actions scheduled on both the Main \ac{DAO} and any Sub-\ac{DAO}, and directly from the Voice UI / or the custom UI for AN \ac{DAO}.
		
	\end{enumerate}
	
	\sectionitem{Separation of Powers}
	\begin{enumerate}
		\item The same members shall not be part of more than one of the following: the Executive Sub-\ac{DAO}, Compliance Sub-\ac{DAO}, Tech Committee, and \glspl{Guardian} in \gls{AragonCourt} mainnet.
		\item When a decentralised identity solution is integrated into the \ac{DAO}, members of these committees shall prove their unique identity using said solution.
	\end{enumerate}

	\sectionitem{Formation of new Sub-DAOs}
	\begin{enumerate}
		\item The Main \ac{DAO} can deploy a new Sub-\ac{DAO}, be given control over an already deployed \ac{DAO}, dissolve an existing Sub-\ac{DAO}, or spin-off a Sub-\ac{DAO} as an independent \ac{DAO} at any point in time.
		\item Each Sub-\ac{DAO} shall have its own operating agreement outlining at a minimum the responsibilities of its members which shall be listed as an Appendix to the \gls{AragonNetworkCharter}.
		\item In case of conflict between agreements, The \gls{AragonNetworkCharter} shall prevail over any Sub-\ac{DAO} operating agreement.
	\end{enumerate}

	\sectionitem{Treasuries \& Permissions}
	\begin{enumerate}
			
		\item \textbf{Investment Vault:} The reserve treasury of the \gls{AragonNetwork}.
		Used to fund Sub-\acp{DAO} and/or to obtain yield.
		\begin{enumerate}
			\item Any \ac{ANT} Holder can make proposals for Financial Proposals as described in the Financial Proposals section of this \ac{AGP} process document.
		\end{enumerate}
		
		\item \textbf{Operations Vault:} The main treasury of the \gls{AragonNetwork}.
		Used to fund operations and make strategic fundings.
		\begin{enumerate}
			\item Any \ac{ANT} Holder can make proposals for Financial Proposals as
			described in the Financial Proposals section of this \ac{AGP} process
			document.
			\item And any wallet with permissions for the Executive \ac{DAO} (i.e. any member of the Executive \ac{DAO}) can program and delete Financial Actions in the Executive Sub-\ac{DAO} as per the Executive Sub-\ac{DAO} operating agreement.
		\end{enumerate}

		\item \textbf{Additional Treasuries}
		\begin{enumerate}
			\item Sub-\acp{DAO} are not allowed to create additional treasuries under the exclusive control of the Sub-\ac{DAO} (and outside of the control of the Main \ac{DAO}) unless approved by a majority vote of \glspl{ANTHolder}.
			\item The Main \ac{DAO} may create additional treasuries through a majority vote of \glspl{ANTHolder}.
		\end{enumerate}  
	
	\end{enumerate}

	\sectionitem{Proposals}
	\begin{enumerate}
		
		\item Any \ac{ANTHolder} can create a Proposal in the Main \ac{DAO}.
		\item Any Sub-\ac{DAO} may remove a scheduled proposal at any time (e.g. Veto right) should they have the power to do so as per their operating agreement and as per this Charter.
		\item All proposals must comply with the Requirements for Proposals \& the specific format and process for the type of proposal as follows:
		
		\subsectionitem{The Requirements for Proposals}\textbf{:}
		\begin{enumerate}
			\item \textbf{Public deliberation:} all proposals must be shared during the public deliberation phase in the Aragon Forum and linked in the Aragon Discord Server, unless these services are unavailable.
			\item \textbf{Voting period:} the vote on \gls{AragonVoice} must be at least 7 days.
			\item \textbf{Scheduling:} once approved, proposals are automatically scheduled for execution 5 days after the end of the vote.
			\item \textbf{Collateral:} the proposer must put 50 \ac{ANT} as collateral during the voting period.
			This collateral might be slashed if the proposal is challenged in \gls{AragonCourt} and ruled to violate any provision of this Charter.
		\end{enumerate}
	
		\subsectionitem{Types of Proposals}\textbf{:}
		\begin{enumerate}		
			\item \textbf{Financial Proposals:} any type of financial transfer of any asset that has a monetary value from the Main \ac{DAO} or any Sub-\ac{DAO}.
			\item \textbf{Elections Proposals:} any type of proposal that gives/removes wallet permissions from the Sub-\acp{DAO}
			\item \textbf{Other Proposals:} any other type of proposal (including Metagovernance Proposals, code submissions, parameter changes, etc).
		\end{enumerate}

		\subsectionitem{Process for Financial Proposals and/or Other Proposals}\textbf{:} in sequential order:
	
		\begin{enumerate}
			
			\item \textbf{Public deliberation phase:} A post with the draft of the proposal is posted in the Aragon forum for a minimum of 7 days and maximum of 14 days with the format:
			\begin{enumerate}
				\item Title of Proposal (in the format ``Financial Proposal:[title]'')
				\item Description of the Action
				\item Description of why the author believes it will help to increase the number of Active Aragon \acp{DAO}
				
				\begin{enumerate}
					\item By default, all Financial Proposals involving a deliverable should use an Escrow (see Additional Definitions).
					In cases where the use of an Escrow is omitted, a justification must be included in the Description.
					\item Greet.me shall be the default Escrow provider for Financial Proposals.
				\end{enumerate}
			
				\item And suggested optional: ETH Wallet address of the author(s) and/or other identifiers
					
			\end{enumerate}

			\item \textbf{Voting:} the proposal (or a revised version of the proposal incorporating the community’s feedback) is posted for a vote on \gls{AragonVoice}.
			\begin{enumerate}
				\item For calculating voting power, 1 \ac{ANT} token = 1 Vote.
			\end{enumerate}
		
			\item \textbf{Approval:} a proposal is deemed approved and scheduled for execution if the following conditions are reached:
			\begin{enumerate}
				\item Quorum: a minimum of five thousand (5,000) \ac{ANT} has been used to vote.
				\item  Support: The vote shall be deemed as “passed” with a simple majority (>50\%) of the participating \ac{ANT} having voted in favour.
				\item When a proposal includes a code submission, the Tech Committee has up to 14 days to decide and communicate via a post in Aragon Forum whether a proposal will be:
				\begin{enumerate}
					\item Accepted and incorporated.
					\item Submitted to a 3rd party audit to determine its safety (conditional on the Operations Vault having the necessary funds).
					\item Rejected as malicious, technically infeasible, or economically infeasible (if an audit is required and the Operations Vault lacks the necessary funds to cover the costs of the audit).
				\end{enumerate}
			
			\end{enumerate}
		
		\end{enumerate}
	
		\subsectionitem{Process for Elections}\textbf{:} in sequential order:
		\begin{enumerate}
			
			\item \textbf{Public deliberation phase:} A post with the draft of the proposal is posted in the Aragon forum for a minimum of 10 days and maximum of 30 days with the format:
			\begin{enumerate}
				\item \textit{Title of Proposal (in the format “Election:[title]”)}
				\item \textit{Description of the Action (including what permissions will be given/removed for which \ac{DAO}/Sub-\ac{DAO})}
				\item \textit{In the event of the election being proposed before the stipulated in the agreements of the Sub-\ac{DAO}, add a description of why an advanced election is needed}
				\item \textit{And suggested optional: ETH Wallet address of the author(s) and/or other identifiers.}
			\end{enumerate}

			\item \textbf{Sourcing Candidates:} Candidates can be proposed (and/or propose themselves) by replying to the Forum post (only one candidate per post reply) in the following format:
			\begin{enumerate}
				\item \textit{Identifier: their ETH wallet address (and optional their name, discord handle, twitter handle, and other identifiers)}
				\item \textit{Rationale: description of why they are an ideal candidate for the position}
			\end{enumerate}

			\item \textbf{Voting:} top 10 candidates with the highest number of upvotes in the Aragon Forum will be put forward to a vote using \gls{AragonVoice}.
			\begin{enumerate}
				\item In the event of the election being proposed before the frequency stipulated in the operating agreement of the Sub-\ac{DAO} (if any), the vote must also include the option to keep the current permissions even if the current wallets were not amongst the top 10 candidates.
			\end{enumerate}

			\item \textbf{Approval \& Execution:}
			\begin{enumerate}
				\item In the exceptional case that two proposed users gain the same number of votes, the winner will be the candidate who reached the tieing number of votes first.
				\begin{enumerate}
					\item E.g. Candidate A and B both tied at 7 votes, Candidate A wins because it reached 7 votes a day before Candidate B.
					If votes are submitted in the same block, repeat the vote.
				\end{enumerate}
			
			\end{enumerate}
		
		\end{enumerate}
	
	\end{enumerate}
	
	\sectionitem{Disputes}
	\begin{enumerate}
		\item Disputes between members that can not be addressed through facilitation or mediation, and disputes related to proposals shall be resolved using \gls{AragonCourt}.
		\item The losing party shall reimburse the winning party for any \gls{AragonCourt} fees incurred by the winning party. Failing that, the Executive \ac{DAO} shall reimburse said fees.
	\end{enumerate}

\end{enumerate}