% SPDX-License-Identifier: CC-BY-SA-4.0

\title{The Aragon Manifesto: A pledge to fight for freedom}
% SPDX-License-Identifier: CC-BY-SA-4.0

\documentclass[
	10pt,
	a4paper,		
	numbers=noenddot,
	parskip=full
]{scrartcl}

%!TEX root = network-dao-charter.tex
% SPDX-License-Identifier: CC-BY-SA-4.0

% Additional Title Field
\usepackage{titling}
\makeatletter
\def\@subtitle{\@latex@warning@no@line{No Version Information Specified}}
\def\subtitle#1{\gdef\@subtitle{#1}}

% Page Geometry
\usepackage[	
	top=2.5cm,
	bottom=2.5cm,
	right=2.5cm,
	left=2.5cm
]{geometry}
\usepackage{booktabs}

% Coloring, Graphics, and Hyperlinks
\usepackage{xcolor,graphicx}
\definecolor{Aragon-Blue}{rgb}{0.2353,0.5412,1}
\usepackage[
	breaklinks=true,
	linktocpage=true,
	colorlinks=true, 
	menucolor={Aragon-Blue},
	linkcolor={black},
	citecolor={Aragon-Blue},
	urlcolor={Aragon-Blue}
]{hyperref}

% Git Version Information
\usepackage[local]{gitinfo2}

% Acronyms and Glossaries
\usepackage[
	acronyms,
	shortcuts,
	nopostdot,
	nogroupskip,
	nonumberlist,
	toc
]{glossaries-extra}
\setabbreviationstyle[acronym]{long-short}

\newglossary[klg]{key}{kld}{kdn}{Key Definitions}
\newglossary[llg]{add}{lld}{ldn}{Additional Definitions}
\makenoidxglossaries

% Custom Fonts
\usepackage{mathspec}
\setmainfont[
	Path			= ./fonts/Overpass/,
	Extension		= .ttf,
	UprightFont		= *-Regular,
	BoldFont		= *-Bold,
	ItalicFont		= *-Italic,
	BoldItalicFont	= *-BoldItalic
]{Overpass}
\setsansfont[
	Path			= ./fonts/Overpass/,
	Extension		= .ttf,
	UprightFont		= *-Regular,
	BoldFont		= *-Bold,
	ItalicFont		= *-Italic,
	BoldItalicFont	= *-BoldItalic
]{Overpass}
\setmathfont[
	Path			= ./fonts/RobotoMono/,
	Extension		= .ttf,
	UprightFont		= *-Regular,
	BoldFont		= *-Bold,
	ItalicFont		= *-Italic,
	BoldItalicFont	= *-BoldItalic
]{RobotoMono}
\setmonofont[
	Path			= ./fonts/RobotoMono/,
	Extension		= .ttf,
	UprightFont		= *-Regular,
	BoldFont		= *-Bold,
	ItalicFont		= *-Italic,
	BoldItalicFont	= *-BoldItalic
]{RobotoMono}

% Deeper Enumerations
\usepackage{enumitem}
\setlist{
	nosep,
	topsep=4pt,
	parsep=0.5\parskip
}
\setlistdepth{5}
\renewlist{enumerate}{enumerate}{5}
\setlist[enumerate,1]{label=\arabic*.}
\setlist[enumerate,2]{label=\alph*.}
\setlist[enumerate,3]{label=\roman*.}
\setlist[enumerate,4]{label=\arabic*.}
\setlist[enumerate,5]{label=\alph*.}

%!TEX root = network-dao-charter.tex
% SPDX-License-Identifier: CC-BY-SA-4.0

% General
\newcommand{\antTokenAddr}{\ethaddress{0xa117000000f279d81a1d3cc75430faa017fa5a2e}}
\newcommand{\aragonCourtAddr}{\ethaddress{0xfb072baa713b01ce944a0515c3e1e98170977daf}}

% Aragon Network Main DAO
\newcommand{\mainDaoAddr}{\ethaddress{0x9c1d24318966793a68e6005eb6b27edace3f28b8}}
\newcommand{\mainDaoTokenAddr}{\antTokenAddr}

% Executive Sub-DAO
\newcommand{\execSubDaoAddr}{\ethaddress{0xe0caf0349cff049f610c3f04f5d7889f58d98d8c}}
\newcommand{\execSubDaoTokenAddr}{\ethaddress{0x613a126c20632c99afd01b044fe13e97b76eeb5a}}
% Election: https://voice.aragon.org/processes/#/0x87cd66fd8d890849a7480a5460d5d2902bed9e30ff1e229c7c548b0cddae58cf
\newcommand{\execSubDaoMemberAddrI}{\ethaddress{0x8B580433568E521ad351b92b98150c0C65ce69B7}}
\newcommand{\execSubDaoMemberAddrII}{\ethaddress{0x3E31155a1c17c9F85e74828447aec412090a4622}}
\newcommand{\execSubDaoMemberAddrIII}{\ethaddress{0x1E8eE48D0621289297693fC98914DA2EfDcE1477}}

% Tech Sub-DAO
\newcommand{\techSubDaoAddr}{-}
\newcommand{\techSubDaoTokenAddr}{0x731B540B83292734F866fF1850532DF1D7A1F80e}
% Election: https://voice.aragon.org/processes/#/0xfc50351d7c1bbd0ab5f4ad17ee839a41a49e7fb4a5ebd47fde2c51b208372193
\newcommand{\techSubDaoMemberAddrI}{\ethaddress{0x806A940E1C431aa36077c4f6fB3e7d36CEF2a9A7}}
\newcommand{\techSubDaoMemberAddrII}{\ethaddress{0xDd9d481BE2ff4957f8Db967F8b301d23bCd4aE64}}
\newcommand{\techSubDaoMemberAddrIII}{\ethaddress{0xCE3d8791c1bdaCc6b8e1a52B4E6aC140F8a2C8c3}}

% Compliance Sub-DAO
\newcommand{\cmplSubDaoAddr}{\ethaddress{0xbe39e9cb1daa8ee8838d6a93d360f7ea7b8373c2}}
\newcommand{\cmplSubDaoTokenAddr}{\ethaddress{0x8aa971084ed42fc3452d34c5aec4878c28dd7cd0}}
% Election: https://voice.aragon.org/processes/#/0xea559447e36ba9ee53e2b0bd8441e2cc4f6f50b69f98cf8702324fe6ca2cb82b)
\newcommand{\cmplSubDaoMemberAddrI}{\ethaddress{0x285fdc3cbf953715ab925ca82ae1e2087de6509a}}
\newcommand{\cmplSubDaoMemberAddrII}{\ethaddress{0x9c05Ea2B1f6BcAfF36Eae0574123D4CC092C6B5C}}
\newcommand{\cmplSubDaoMemberAddrIII}{\ethaddress{0x75cD58A01883C6A115A1293C4Dd4CE67D11928f0}}

\subtitle{\gitDescribe}
\author{Aragon Network}
\date{\today}

%!TEX root = network-dao-charter.tex
% SPDX-License-Identifier: CC-BY-SA-4.0

% Acronyms and definitions have to be defined here so that the can be used in the text later.

%%%%%%%%%%
% Acronyms
%%%%%%%%%%
\newacronym{AGP}{AGP}{Aragon Governance Proposal}
\newacronym{AN}{AN}{\glslink{AragonNetwork}{Aragon Network}}\glsunset{AN}
\newacronym{ANT}{ANT}{\glslink{AragonNetwork}{Aragon Network} \glslink{AragonNetworkToken}{Token}}\glsunset{ANT}
\newacronym{DAO}{DAO}{\glslink{DecentralizedAutonomousOrganization}{Decentralized Autonomous Organization}}\glsunset{DAO}
\newacronym{ENS}{ENS}{\glslink{EthereumNameService}{Ethereum Name Service}}\glsunset{ENS}
\newacronym{IPFS}{IPFS}{\glslink{IPFSNetwork}{InterPlanetary File System}}

%%%%%%%%%%%%%%%%%
% Key definitions
%%%%%%%%%%%%%%%%%
\newglossaryentry{ActiveDAOs}{
	type=key,
	name={Active Aragon \acp{DAO}},
	description={
		\acp{DAO} created using the smart contract provided by the Aragon Network, that have enacted at least 1 decision in the past month.
	}
}

\newglossaryentry{AragonNetwork}{
	type=key,
	name={\acl{AN}},
	description={
		a network of \glspl{ANTHolder} and the organisations and technology created by them to pursue the goals and live the values defined in the \href{https://github.com/aragon/AGPs/blob/master/AGPs/AGP-0.md}{Aragon Manifesto}. 
		Including but not limited to: The Aragon Network \ac{DAO}, \glspl{ANTHolder}, their affiliates, employees, contributors, licensors and service providers, and their respective officers, directors, employees, contractors, agents, licensors, suppliers, parent companies, successors, subsidiaries, affiliates, agents, representatives.
	}
}

\newglossaryentry{AragonNetworkCharter}{
	type=key,
	name=\acl{AN} Charter,
	description={the whole of this document (``this Charter''), which consists of the human-interpreted rules of the \gls{AragonNetwork} and is subdivided in the following human-readable documents:
		\begin{itemize}[noitemsep]
			\item \hyperref[chap:ANDAOAgreement]{\textbf{The Aragon Network DAO Agreement}}: the responsibilities and rights of \glspl{ANTHolder}.
			\item \hyperref[chap:AragonManifesto]{\textbf{The Aragon Manifesto}}: the values and the mission of the \gls{AragonNetwork}.
			\item \hyperref[chap:CommunityGuidelines]{\textbf{The Community Guidelines}}: the norms of behaviour between \glspl{ANTHolder}.
			\item \hyperref[chap:AGPProcess]{\textbf{The \ac{AGP} Process}}: the process used by \glspl{ANTHolder} to govern the \gls{AragonNetworkDAO}
		\end{itemize}
	}
}		

\newglossaryentry{AragonNetworkDAO}{
	type=key,
	name={\acl{AN} \ac{DAO}},
	description={
		a \ac{DAO} (an open coordination protocol based on open blockchain technology) that aims to facilitate the governance of the Aragon Network’s tools, infrastructure and other common resources. The Aragon Network \ac{DAO} is governed by \glspl{ANTHolder} according to the \ac{AGP} Process and includes one or more Sub-\acp{DAO} also binded by this Charter.
	}
}

\newglossaryentry{AragonNetworkToken}{
	type=key,
	name={\ac{ANT}},
	description={
		the native token of the \gls{AragonNetwork} - a transferable ERC-20 token with the contract address: \ethaddress{0xa117000000f279d81a1d3cc75430faa017fa5a2e}.
		ANT is used for the governance of the Aragon Network DAO and other forms of participation in the Aragon Network.
	}
}

\newglossaryentry{ANTHolder}{
	type=key,
	name={\ac{ANT} Holder},
	description={
		any entity that owns or controls an Ethereum account that controls a positive balance of \ac{ANT}.
	}
}

\newglossaryentry{AragonCourt}{
	type=key,
	name={Aragon Court},
	description={
		a smart contract-based dispute resolution system governed by \glspl{ANTHolder}.
	}
}

\newglossaryentry{AragonVoice}{
	type=key,
	name={Aragon Voice},
	description={A token-based voting client with universally verifiable results and deterministic execution.
	}
}

\newglossaryentry{CommunityMember}{
	type=key,
	name={Community Member},
	description={
		any \ac{ANT} Holder participating in the communication platforms, including Aragon’s Discord server, Aragon’s Telegram group, Aragon Forum, Aragon Network \ac{DAO}, online and offline gatherings, and/or other groups and platforms of the Aragon Network.
	}
}

\newglossaryentry{FinancialActions}{
	type=key, 
	name={Financial Actions},
	description={
		any on-chain action that transfers, exchanges, or stakes tokens held by the \ac{AN} \ac{DAO} or Sub-\acp{DAO}.
	}
}


%%%%%%%%%%%%%%%%%%%%%%%%
% Additional Definitions
%%%%%%%%%%%%%%%%%%%%%%%%

\newglossaryentry{AragonNetworkCommunityPlatforms}{
	type=add,
	name={Aragon Network Community Platforms},
	description={
		the platforms used for regular communication by the \glspl{ANTHolder} and managed by either the Aragon Association or the \ac{AN} \ac{DAO}. They include but are not restricted to Aragon’s Discord Server, Aragon’s Forum, Aragon’s GitHub repository, and online and offline events organised under the Aragon brand.
	}
}

\newglossaryentry{AragonNetworkOrganization}{
	type=add,
	name={Aragon Network organization},
	description={
		the Aragon organization addressed by the \ac{ENS} name\\ \ethaddress{network.aragonid.eth}, encompassing all apps installed on this organization.
	}
}

\newglossaryentry{DecentralizedAutonomousOrganization}{
	type=add,
	name=\glslink{DAO}{DAO},
	description={
		an open-source blockchain protocol governed by a set of rules, with fluid affiliation or membership through participation, and that automatically execute certain actions without the need for intermediaries.
	}
}

\newglossaryentry{EthereumNameService}{
	type=add,
	name={\ac{ENS}},
	description={
		a smart contract-based naming system. \ac{ENS} names are registered using the registry deployed at the Ethereum address \ethaddress{0x00000000000C2E074eC69A0dFb2997BA6C7d2e1e}.
	}
}

\newglossaryentry{Escrow}{
	type=add,
	name={Escrow},
	description={
		a contractual arrangement in which a third party (the stakeholder or escrow agent) receives and disburses money or property for the primary transacting parties, with the disbursement dependent on conditions agreed to by the transacting parties.
	}
}

\newglossaryentry{EthereumBlockchain}{
	type=add,
	name={Ethereum blockchain},
	description={
		The blockchain referred to as ``Ethereum'' by the owner of the Ethereum trademark, the Ethereum Foundation (Stiftung Ethereum).
	}
}

\newglossaryentry{Guardian}{
	type=add,
	name={Guardian},
	description={
		natural person (represented by a wallet address) that is eligible to be drafted to rule upon disputes in \gls{AragonCourt}. 
		If a guardian is summoned, it must vote to allow or block the action being disputed, and can be rewarded (earn tokens) or punished (lose tokens) depending if they voted with the majority (default interpretation) or against it. 
		The protocol is designed so that Guardians do not know each other and do not have contact when they are summoned.
	}
}


\newglossaryentry{IPFSNetwork}{
	type=add,
	name={\ac{IPFS} network},
	description={
		A peer-to-peer hypermedia protocol, whose project website is located at \href{https://ipfs.io}{https://ipfs.io}. 
		The public network makes files accessible to any internet-connected device running a compatible implementation of the \ac{IPFS} software.
	}
}

\newglossaryentry{Metagovernance}{
	type=add,
	name={Metagovernance (``Metagov'')},
	description={
		refers to the process(es) to change the governance processes of the \gls{AragonNetworkDAO}.
	}
}

\newglossaryentry{NetworkContract}{
	type=add,
	name={Network contract},
	description={
		The computer-interpreted source code of a smart contract governed by \glspl{ANTHolder}.
	}
}

\newglossaryentry{NetworkParameters}{
	type=add,
	name={Network parameters},
	description={
		The current parameters of smart contracts governed by \glspl{ANTHolder}, which can be modified without amending the source code of the smart contract itself.
	}
}
 
\newglossaryentry{Proposal}{
	type=add,
	name={Proposal},
	description={
		any suggested amendment to the Network contracts, Network parameters, the Agreements of the \gls{AragonNetwork} (i.e. \gls{AragonNetworkCharter} and composing agreements), and Agreements of any Sub-\acp{DAO}.
	}
}

\newglossaryentry{SmartContract}{
	type=add,
	name={Smart contract},
	description={
		a software program deployed to a blockchain.
	}
}

\newglossaryentry{OffChain}{
	type=add,
	name={Off-Chain},
	description={
		refers to any action or data entry that is not recorded on a blockchain or other type of distributed ledger where the smart contract (or version of it) is deployed or connected to so as to have verifiable and tamper-proof records.
	}
}

\newglossaryentry{OnChain}{
	type=add,
	name={On-Chain},
	description={
		refers to any action or data entry that is recorded on a blockchain or other type of distributed ledger where the smart contract (or version of it) is deployed or connected to so as to have verifiable and tamper-proof records.
	}
}

\newglossaryentry{OptimisticGovernance}{
	type=add,
	name={Optimistic Governance},
	description={
		the process of decisions being enacted with a time delay which allows time for a decision to be challenged. If a decision is not disputed then the change is enacted at the end of the time delay. This allows for much faster decision making and strategy decisions to be made.
	}
}





\begin{document}
\mytitle

\textbf{We believe the fate of humanity will be decided at the frontier of technological innovation.} 
We will either see technology lead to a more free, open, and fair society or reinforce a global regime of centralized control, surveillance, and oppression. Our fear is that without a global, conscious, and concerted effort, the outlook is incredibly bleak.

The Internet has opened the doors for universal, cross-border, and non-violent collaborative efforts to \textbf{fight for our freedom}.

However, the Internet has also opened the doors for global surveillance and manipulation.

We believe humankind \textbf{should use technology as a liberating tool} to unleash all the goodwill and creativity of our species, rather than as a tool to enslave and take advantage of one another.

Thus, Aragon is a fight for freedom. Aragon empowers freedom by creating liberating tools that leverage decentralized technologies.

\textbf{Decentralized technologies provide users unparalleled power} to transact and interact with a level of security never seen before. Thanks to cryptography and economic incentives, users can now own truly sovereign assets, create fully sovereign entities, and build truly sovereign identities. They solidify freedoms that cannot be taken away, not even by actors with sizeable re- sources. This tectonic shift requires a new method for organizing these sovereign individuals: \textbf{decentralized organizations}.

For the first time in history, thanks to blockchain technology and smart contracts, we can now create fully decentralized organizations, which are truly autonomous and unstoppable.

Decentralized organizations change our relationship with \textbf{governance}: from something that is imposed upon us by others, into something we choose to opt into. 
Where we are equally serving and served, rather than just serving.

Building tools to create and manage decentralized organizations will unleash a cambrian explosion of new governance forms, and the competition among them will raise the bar globally.

It will finally allow us to experiment with governance at the speed of software and learn through the empathy of a collective design approach.

Instead of complaining about how badly incentives are set in the world and how poorly resources are allocated, we will have the power to create systems that better align incentives and distribute resources. This is the enlightenment of the century.

Sovereign individuals will be able to freely express themselves and transact with each other \textbf{without any kind of intermediary} exercising their unjustified power and oppressing them.

This may be one of the most important revolutions humanity has ever faced.
We need to be careful about how we go about it.
Technology can be a double-edged sword.
We need to explicitly set out \textbf{the values that Aragon stands for}, in order for our community to always \textbf{uphold and defend them}.

\textbf{We are committed to building organizational forms that defend self-sovereignty — where a user can always exercise choice, either by participating or exiting.}

In today's world, our governing power over society's common goods is negligible and inaccessible.

Even for exercising small changes over your environment (your city, state or corporation), you might have to go through cumbersome and opaque bureaucratic processes.

This happens because the coordination costs were high in the old world.
But now, thanks to the Internet and decentralized governance, we can, and must, engineer systems to empower the people's voice to their fullest potential.

Also, if you dislike those broken governance processes, \textbf{you cannot easily exit from them}.
Some organizations own your identity (governments), some others own your data (Facebook), and some others even own the path for you to exit (borders), restricting your choice of leaving.

\textbf{We must build organizational structures that allow users and other stakeholders to exit with minimal friction} if they fundamentally disagree with the governance of those structures.

\textbf{We are committed to creating collaboration mechanisms in which violence is not only disincentivized, but impossible.}

Choices can only be made freely in the absence of coercion and extortion.
\textbf{Systems should never use violence} as a means to incentivize or disincentivize human behavior.

With cryptonetworks, we can create environments where others cannot even tell where individuals are located or what they look like.
We must make violence in these cryptonetworks impossible by protecting users' privacy with cryptography.

\textbf{We are committed to decentralizing power in order to dismantle unjustified power — which usually springs from centralization.}

Power has a natural tendency to reinforce itself and become corrupt.
If power becomes centralized, it doesn't have to answer to anyone but itself.
Decentralizing power is essential to minimizing corruption over time.

We must strive to create systems in which a large number of diverse stakeholders have a say, in order for common goods to be responsibly governed by their communities and the relationships between stakeholders based on egalitarian values.

\textbf{We are committed to the creation of long-term value versus short-term profit — which in turn, advances regeneration}.

We want these values to last.
\textbf{We are in this for the long run.} 
In order to make a lasting impact and disrupt existing power structures, \textbf{we must create systems that provide regenerative economic value to their participants}.

We must honor and encourage the communities that sustain Aragon itself, and we must do so by rewarding those who defend the values outlined in this manifesto.

\textbf{We are committed to a world in which every person can participate in these new organizational structures.}

We seek to use technology to lift people from oppression.
To be successful, \textbf{we must keep our products open, understandable, and easy to use} for everyone.

\end{document}