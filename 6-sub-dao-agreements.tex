% SPDX-License-Identifier: CC-BY-SA-4.0

\title{Aragon Network Sub-DAO Agreements}
% SPDX-License-Identifier: CC-BY-SA-4.0

\documentclass[
	10pt,
	a4paper,		
	numbers=noenddot,
	parskip=full
]{scrartcl}

%!TEX root = network-dao-charter.tex
% SPDX-License-Identifier: CC-BY-SA-4.0

% Additional Title Field
\usepackage{titling}
\makeatletter
\def\@subtitle{\@latex@warning@no@line{No Version Information Specified}}
\def\subtitle#1{\gdef\@subtitle{#1}}

% Page Geometry
\usepackage[	
	top=2.5cm,
	bottom=2.5cm,
	right=2.5cm,
	left=2.5cm
]{geometry}
\usepackage{booktabs}

% Coloring, Graphics, and Hyperlinks
\usepackage{xcolor,graphicx}
\definecolor{Aragon-Blue}{rgb}{0.2353,0.5412,1}
\usepackage[
	breaklinks=true,
	linktocpage=true,
	colorlinks=true, 
	menucolor={Aragon-Blue},
	linkcolor={black},
	citecolor={Aragon-Blue},
	urlcolor={Aragon-Blue}
]{hyperref}

% Git Version Information
\usepackage[local]{gitinfo2}

% Acronyms and Glossaries
\usepackage[
	acronyms,
	shortcuts,
	nopostdot,
	nogroupskip,
	nonumberlist,
	toc
]{glossaries-extra}
\setabbreviationstyle[acronym]{long-short}

\newglossary[klg]{key}{kld}{kdn}{Key Definitions}
\newglossary[llg]{add}{lld}{ldn}{Additional Definitions}
\makenoidxglossaries

% Custom Fonts
\usepackage{mathspec}
\setmainfont[
	Path			= ./fonts/Overpass/,
	Extension		= .ttf,
	UprightFont		= *-Regular,
	BoldFont		= *-Bold,
	ItalicFont		= *-Italic,
	BoldItalicFont	= *-BoldItalic
]{Overpass}
\setsansfont[
	Path			= ./fonts/Overpass/,
	Extension		= .ttf,
	UprightFont		= *-Regular,
	BoldFont		= *-Bold,
	ItalicFont		= *-Italic,
	BoldItalicFont	= *-BoldItalic
]{Overpass}
\setmathfont[
	Path			= ./fonts/RobotoMono/,
	Extension		= .ttf,
	UprightFont		= *-Regular,
	BoldFont		= *-Bold,
	ItalicFont		= *-Italic,
	BoldItalicFont	= *-BoldItalic
]{RobotoMono}
\setmonofont[
	Path			= ./fonts/RobotoMono/,
	Extension		= .ttf,
	UprightFont		= *-Regular,
	BoldFont		= *-Bold,
	ItalicFont		= *-Italic,
	BoldItalicFont	= *-BoldItalic
]{RobotoMono}

% Deeper Enumerations
\usepackage{enumitem}
\setlist{
	nosep,
	topsep=4pt,
	parsep=0.5\parskip
}
\setlistdepth{5}
\renewlist{enumerate}{enumerate}{5}
\setlist[enumerate,1]{label=\arabic*.}
\setlist[enumerate,2]{label=\alph*.}
\setlist[enumerate,3]{label=\roman*.}
\setlist[enumerate,4]{label=\arabic*.}
\setlist[enumerate,5]{label=\alph*.}

%!TEX root = network-dao-charter.tex
% SPDX-License-Identifier: CC-BY-SA-4.0

% General
\newcommand{\antTokenAddr}{\ethaddress{0xa117000000f279d81a1d3cc75430faa017fa5a2e}}
\newcommand{\aragonCourtAddr}{\ethaddress{0xfb072baa713b01ce944a0515c3e1e98170977daf}}

% Aragon Network Main DAO
\newcommand{\mainDaoAddr}{\ethaddress{0x9c1d24318966793a68e6005eb6b27edace3f28b8}}
\newcommand{\mainDaoTokenAddr}{\antTokenAddr}

% Executive Sub-DAO
\newcommand{\execSubDaoAddr}{\ethaddress{0xe0caf0349cff049f610c3f04f5d7889f58d98d8c}}
\newcommand{\execSubDaoTokenAddr}{\ethaddress{0x613a126c20632c99afd01b044fe13e97b76eeb5a}}
% Election: https://voice.aragon.org/processes/#/0x87cd66fd8d890849a7480a5460d5d2902bed9e30ff1e229c7c548b0cddae58cf
\newcommand{\execSubDaoMemberAddrI}{\ethaddress{0x8B580433568E521ad351b92b98150c0C65ce69B7}}
\newcommand{\execSubDaoMemberAddrII}{\ethaddress{0x3E31155a1c17c9F85e74828447aec412090a4622}}
\newcommand{\execSubDaoMemberAddrIII}{\ethaddress{0x1E8eE48D0621289297693fC98914DA2EfDcE1477}}

% Tech Sub-DAO
\newcommand{\techSubDaoAddr}{-}
\newcommand{\techSubDaoTokenAddr}{0x731B540B83292734F866fF1850532DF1D7A1F80e}
% Election: https://voice.aragon.org/processes/#/0xfc50351d7c1bbd0ab5f4ad17ee839a41a49e7fb4a5ebd47fde2c51b208372193
\newcommand{\techSubDaoMemberAddrI}{\ethaddress{0x806A940E1C431aa36077c4f6fB3e7d36CEF2a9A7}}
\newcommand{\techSubDaoMemberAddrII}{\ethaddress{0xDd9d481BE2ff4957f8Db967F8b301d23bCd4aE64}}
\newcommand{\techSubDaoMemberAddrIII}{\ethaddress{0xCE3d8791c1bdaCc6b8e1a52B4E6aC140F8a2C8c3}}

% Compliance Sub-DAO
\newcommand{\cmplSubDaoAddr}{\ethaddress{0xbe39e9cb1daa8ee8838d6a93d360f7ea7b8373c2}}
\newcommand{\cmplSubDaoTokenAddr}{\ethaddress{0x8aa971084ed42fc3452d34c5aec4878c28dd7cd0}}
% Election: https://voice.aragon.org/processes/#/0xea559447e36ba9ee53e2b0bd8441e2cc4f6f50b69f98cf8702324fe6ca2cb82b)
\newcommand{\cmplSubDaoMemberAddrI}{\ethaddress{0x285fdc3cbf953715ab925ca82ae1e2087de6509a}}
\newcommand{\cmplSubDaoMemberAddrII}{\ethaddress{0x9c05Ea2B1f6BcAfF36Eae0574123D4CC092C6B5C}}
\newcommand{\cmplSubDaoMemberAddrIII}{\ethaddress{0x75cD58A01883C6A115A1293C4Dd4CE67D11928f0}}

\subtitle{\gitDescribe}
\author{Aragon Network}
\date{\today}

%!TEX root = network-dao-charter.tex
% SPDX-License-Identifier: CC-BY-SA-4.0

% Acronyms and definitions have to be defined here so that the can be used in the text later.

%%%%%%%%%%
% Acronyms
%%%%%%%%%%
\newacronym{AGP}{AGP}{Aragon Governance Proposal}
\newacronym{AN}{AN}{\glslink{AragonNetwork}{Aragon Network}}\glsunset{AN}
\newacronym{ANT}{ANT}{\glslink{AragonNetwork}{Aragon Network} \glslink{AragonNetworkToken}{Token}}\glsunset{ANT}
\newacronym{DAO}{DAO}{\glslink{DecentralizedAutonomousOrganization}{Decentralized Autonomous Organization}}\glsunset{DAO}
\newacronym{ENS}{ENS}{\glslink{EthereumNameService}{Ethereum Name Service}}\glsunset{ENS}
\newacronym{IPFS}{IPFS}{\glslink{IPFSNetwork}{InterPlanetary File System}}

%%%%%%%%%%%%%%%%%
% Key definitions
%%%%%%%%%%%%%%%%%
\newglossaryentry{ActiveDAOs}{
	type=key,
	name={Active Aragon \acp{DAO}},
	description={
		\acp{DAO} created using the smart contract provided by the Aragon Network, that have enacted at least 1 decision in the past month.
	}
}

\newglossaryentry{AragonNetwork}{
	type=key,
	name={\acl{AN}},
	description={
		a network of \glspl{ANTHolder} and the organisations and technology created by them to pursue the goals and live the values defined in the \href{https://github.com/aragon/AGPs/blob/master/AGPs/AGP-0.md}{Aragon Manifesto}. 
		Including but not limited to: The Aragon Network \ac{DAO}, \glspl{ANTHolder}, their affiliates, employees, contributors, licensors and service providers, and their respective officers, directors, employees, contractors, agents, licensors, suppliers, parent companies, successors, subsidiaries, affiliates, agents, representatives.
	}
}

\newglossaryentry{AragonNetworkCharter}{
	type=key,
	name=\acl{AN} Charter,
	description={the whole of this document (``this Charter''), which consists of the human-interpreted rules of the \gls{AragonNetwork} and is subdivided in the following human-readable documents:
		\begin{itemize}[noitemsep]
			\item \hyperref[chap:ANDAOAgreement]{\textbf{The Aragon Network DAO Agreement}}: the responsibilities and rights of \glspl{ANTHolder}.
			\item \hyperref[chap:AragonManifesto]{\textbf{The Aragon Manifesto}}: the values and the mission of the \gls{AragonNetwork}.
			\item \hyperref[chap:CommunityGuidelines]{\textbf{The Community Guidelines}}: the norms of behaviour between \glspl{ANTHolder}.
			\item \hyperref[chap:AGPProcess]{\textbf{The \ac{AGP} Process}}: the process used by \glspl{ANTHolder} to govern the \gls{AragonNetworkDAO}
		\end{itemize}
	}
}		

\newglossaryentry{AragonNetworkDAO}{
	type=key,
	name={\acl{AN} \ac{DAO}},
	description={
		a \ac{DAO} (an open coordination protocol based on open blockchain technology) that aims to facilitate the governance of the Aragon Network’s tools, infrastructure and other common resources. The Aragon Network \ac{DAO} is governed by \glspl{ANTHolder} according to the \ac{AGP} Process and includes one or more Sub-\acp{DAO} also binded by this Charter.
	}
}

\newglossaryentry{AragonNetworkToken}{
	type=key,
	name={\ac{ANT}},
	description={
		the native token of the \gls{AragonNetwork} - a transferable ERC-20 token with the contract address: \ethaddress{0xa117000000f279d81a1d3cc75430faa017fa5a2e}.
		ANT is used for the governance of the Aragon Network DAO and other forms of participation in the Aragon Network.
	}
}

\newglossaryentry{ANTHolder}{
	type=key,
	name={\ac{ANT} Holder},
	description={
		any entity that owns or controls an Ethereum account that controls a positive balance of \ac{ANT}.
	}
}

\newglossaryentry{AragonCourt}{
	type=key,
	name={Aragon Court},
	description={
		a smart contract-based dispute resolution system governed by \glspl{ANTHolder}.
	}
}

\newglossaryentry{AragonVoice}{
	type=key,
	name={Aragon Voice},
	description={A token-based voting client with universally verifiable results and deterministic execution.
	}
}

\newglossaryentry{CommunityMember}{
	type=key,
	name={Community Member},
	description={
		any \ac{ANT} Holder participating in the communication platforms, including Aragon’s Discord server, Aragon’s Telegram group, Aragon Forum, Aragon Network \ac{DAO}, online and offline gatherings, and/or other groups and platforms of the Aragon Network.
	}
}

\newglossaryentry{FinancialActions}{
	type=key, 
	name={Financial Actions},
	description={
		any on-chain action that transfers, exchanges, or stakes tokens held by the \ac{AN} \ac{DAO} or Sub-\acp{DAO}.
	}
}


%%%%%%%%%%%%%%%%%%%%%%%%
% Additional Definitions
%%%%%%%%%%%%%%%%%%%%%%%%

\newglossaryentry{AragonNetworkCommunityPlatforms}{
	type=add,
	name={Aragon Network Community Platforms},
	description={
		the platforms used for regular communication by the \glspl{ANTHolder} and managed by either the Aragon Association or the \ac{AN} \ac{DAO}. They include but are not restricted to Aragon’s Discord Server, Aragon’s Forum, Aragon’s GitHub repository, and online and offline events organised under the Aragon brand.
	}
}

\newglossaryentry{AragonNetworkOrganization}{
	type=add,
	name={Aragon Network organization},
	description={
		the Aragon organization addressed by the \ac{ENS} name\\ \ethaddress{network.aragonid.eth}, encompassing all apps installed on this organization.
	}
}

\newglossaryentry{DecentralizedAutonomousOrganization}{
	type=add,
	name=\glslink{DAO}{DAO},
	description={
		an open-source blockchain protocol governed by a set of rules, with fluid affiliation or membership through participation, and that automatically execute certain actions without the need for intermediaries.
	}
}

\newglossaryentry{EthereumNameService}{
	type=add,
	name={\ac{ENS}},
	description={
		a smart contract-based naming system. \ac{ENS} names are registered using the registry deployed at the Ethereum address \ethaddress{0x00000000000C2E074eC69A0dFb2997BA6C7d2e1e}.
	}
}

\newglossaryentry{Escrow}{
	type=add,
	name={Escrow},
	description={
		a contractual arrangement in which a third party (the stakeholder or escrow agent) receives and disburses money or property for the primary transacting parties, with the disbursement dependent on conditions agreed to by the transacting parties.
	}
}

\newglossaryentry{EthereumBlockchain}{
	type=add,
	name={Ethereum blockchain},
	description={
		The blockchain referred to as ``Ethereum'' by the owner of the Ethereum trademark, the Ethereum Foundation (Stiftung Ethereum).
	}
}

\newglossaryentry{Guardian}{
	type=add,
	name={Guardian},
	description={
		natural person (represented by a wallet address) that is eligible to be drafted to rule upon disputes in \gls{AragonCourt}. 
		If a guardian is summoned, it must vote to allow or block the action being disputed, and can be rewarded (earn tokens) or punished (lose tokens) depending if they voted with the majority (default interpretation) or against it. 
		The protocol is designed so that Guardians do not know each other and do not have contact when they are summoned.
	}
}


\newglossaryentry{IPFSNetwork}{
	type=add,
	name={\ac{IPFS} network},
	description={
		A peer-to-peer hypermedia protocol, whose project website is located at \href{https://ipfs.io}{https://ipfs.io}. 
		The public network makes files accessible to any internet-connected device running a compatible implementation of the \ac{IPFS} software.
	}
}

\newglossaryentry{Metagovernance}{
	type=add,
	name={Metagovernance (``Metagov'')},
	description={
		refers to the process(es) to change the governance processes of the \gls{AragonNetworkDAO}.
	}
}

\newglossaryentry{NetworkContract}{
	type=add,
	name={Network contract},
	description={
		The computer-interpreted source code of a smart contract governed by \glspl{ANTHolder}.
	}
}

\newglossaryentry{NetworkParameters}{
	type=add,
	name={Network parameters},
	description={
		The current parameters of smart contracts governed by \glspl{ANTHolder}, which can be modified without amending the source code of the smart contract itself.
	}
}
 
\newglossaryentry{Proposal}{
	type=add,
	name={Proposal},
	description={
		any suggested amendment to the Network contracts, Network parameters, the Agreements of the \gls{AragonNetwork} (i.e. \gls{AragonNetworkCharter} and composing agreements), and Agreements of any Sub-\acp{DAO}.
	}
}

\newglossaryentry{SmartContract}{
	type=add,
	name={Smart contract},
	description={
		a software program deployed to a blockchain.
	}
}

\newglossaryentry{OffChain}{
	type=add,
	name={Off-Chain},
	description={
		refers to any action or data entry that is not recorded on a blockchain or other type of distributed ledger where the smart contract (or version of it) is deployed or connected to so as to have verifiable and tamper-proof records.
	}
}

\newglossaryentry{OnChain}{
	type=add,
	name={On-Chain},
	description={
		refers to any action or data entry that is recorded on a blockchain or other type of distributed ledger where the smart contract (or version of it) is deployed or connected to so as to have verifiable and tamper-proof records.
	}
}

\newglossaryentry{OptimisticGovernance}{
	type=add,
	name={Optimistic Governance},
	description={
		the process of decisions being enacted with a time delay which allows time for a decision to be challenged. If a decision is not disputed then the change is enacted at the end of the time delay. This allows for much faster decision making and strategy decisions to be made.
	}
}





\begin{document}
\mytitle

\section{General Provisions}

\subsection{Agreements}
\begin{enumerate}
	\item The Sub-\ac{DAO} is bound primarily to the Aragon Network Charter, and secondarily to its own rules and/or Agreements.
\end{enumerate}


\subsection{Membership}

\begin{enumerate}
	\item The Sub-\ac{DAO} shall initially consist of 3 members.
	\begin{enumerate}
		\item In the first year since the creation of the Sub-\ac{DAO}, 2 of these members will be elected by the Aragon Association and 1 member will be elected through an election proposal in the Main \ac{DAO}.
		\item In year 2, 1 of these members will be elected by the Aragon Association and 2 members will be elected through an election proposal in the Main \ac{DAO}.
		\item In year 3, all members will be elected through an election proposal in the Main \ac{DAO}.
		\item The term of each member shall be of 1 year.
		\item Should multiple Committee members be reelected for more than 3 consecutive terms, the member that has been elected for the longest period of time will be discharged and replaced with a new Committee member.
		The discharged member(s) can be re-elected after 2 years.
		\item Only natural persons can be members of the Sub-\ac{DAO}, verified using a decentralized identity solution, chosen by the Main \ac{DAO}.
	\end{enumerate}
\end{enumerate}


\subsection{Termination of Membership}

\begin{enumerate}
	\item If a member commits a serious breach (as determined by the Main \ac{DAO}) of the Charter, the Main \ac{DAO} may at any time remove said member of a Sub-\ac{DAO}.
	The member may appeal the decision.
	\item All proposals for termination will be carried out by a vote.
	\item Any appeals shall be carried out using \gls{TechAgnosticMechanisms}
\end{enumerate}


\subsection{Compensation \& Expenses}

\begin{enumerate}
	\item Each Committee member of a Sub-\ac{DAO} shall be paid a monthly fee of 200~\ac{ANT} per month.
	\item Committee members shall be reimbursed by the Executive \ac{DAO} of any reasonable expenses incurred in the performance of their duties.
\end{enumerate}


\section{Executive Sub-DAO}

\subsection{Responsibilities}

In particular the Executive \ac{DAO} has the following powers:
\begin{enumerate}
	\item Pay members of other Sub-\acp{DAO}
	\item Make grants to other community members at their discretion, providing:
	\begin{enumerate}
		\item Such transactions are disclosed transparently on the Aragon Forum website.
		\item An Escrow is used to hold funds until the completion of the agreed-upon deliverable.
		\item The deliverable has been fully assessed by the members of the Executive \ac{DAO}.
	\end{enumerate}
	\item Pay suppliers of the Aragon network, providing such transactions are disclosed transparently on the Aragon Forum.
\end{enumerate}

Executive \ac{DAO} members have the following responsibilities:
\begin{enumerate}
	\item Keep an up to date record of their activities and use of funds.
	\item Hold a General Meeting (online or offline) every fortnight and keep a record of the meeting available to the \glspl{ANTHolder}.
	\item Not miss any more than 3 consecutive General Meetings without providing a valid excuse (medical or force majeure) or a public explanation to \glspl{ANTHolder}.
\end{enumerate}


\section{Decision-Making Process} 

The Executive \ac{DAO} uses a lazy-consensus process as follows:
\begin{enumerate}
	
	\item Any member of the Executive \ac{DAO} can schedule an action in the \ac{DAO}.
	\item If another member disagrees or wishes to discuss said action, provided that said action has not been backed by a majority vote of the Executive \ac{DAO} members they may:
	\begin{enumerate}
		\item Speak directly with the member who scheduled the action, preferably through a channel that’s readable asynchronously by the community
		\item and, if needed, cancel the scheduled action to provide enough time for the discussion.
	\end{enumerate}
	
	\item If the discussion between members doesn’t lead to an agreement or can not be scheduled soon enough (as determined subjectively by each member), each member of the executive \ac{DAO} can trigger a vote of the Executive \ac{DAO} members to resolve the dispute through a simple majority.
	The vote must be open for a minimum of 4 days and no longer than 14 days.
	\item The vote’s result will be invalid if:
	\begin{enumerate}
		\item Not all Committee members vote AND the voting period has lasted fewer than 14 days or longer than 30 days.
		For clarity, if all Committee members vote, the vote will be considered valid irrespective of the duration.
		\item Fewer than 50\% of Committee members have voted.
	\end{enumerate}

\end{enumerate}


\section{Tech Committee Sub-DAO}

\subsection{Responsibilities}		

In particular, the Tech Committee \ac{DAO} has the following responsibilities:
\begin{enumerate}		
	\item Review technical proposals in the Main \ac{DAO} and Sub-\acp{DAO} to assess them from a technical risk perspective.
	\item Remove technical proposals in the Main \ac{DAO} and Sub-\acp{DAO} that represent a significant technical risk to the project.
	\item Approve technical proposals they believe would be beneficial to the Aragon project and DO NOT require a 3rd party technical security audit due to being low risk.
	\item Suspend technical proposals they believe would be beneficial to the Aragon project, pending the completion of a 3rd party technical security audit.
	\item When needed, add approved proposals to GitHub and other repositories and merge the code.
	\item Maintaining a list of whitelisted technical security auditors they deem to be sufficiently competent to audit Aragon smart contracts.
\end{enumerate}



\subsection{Decision-Making Process}

\begin{enumerate}
	\item The Tech Committee decides on a simple majority basis through a vote of the Tech Committee members.
	\item The vote’s result will be invalid if:
	\begin{enumerate}
		\item Not all Committee members vote AND the voting period has lasted
		fewer than 14 days or longer than 30 days.
		For clarity, if all Committee members vote, the vote will be considered valid irrespective of the duration.
		\item Fewer than 50\% of Committee members have voted.
	\end{enumerate}
\end{enumerate}


\section{Compliance Committee Sub-DAO}

\subsection{Responsibilities}

In particular the Compliance Committee \ac{DAO} has the following responsibilities:
\begin{enumerate}
	\item Reviewing all proposals in the AN \ac{DAO} and any-sub \ac{DAO} for
	compliance with this Charter and overall legal compliance,
	providing feedback to proposal creators where appropriate.
	\item Remove any proposals they deem to be non-compliant with any
	part of this Charter or illegal.
\end{enumerate}


\subsection{Legal Responsibility}

\begin{enumerate}
	\item The Compliance Committee members assume full legal responsibility for the approval of any illegal, unlawful, criminal or fraudulent proposal.
\end{enumerate}


\subsection{Decision-Making Process}

The Compliance Committee \ac{DAO} operates on an lazy-consensus veto basis as follows:
\begin{enumerate}
	
	\item Any member of the Compliance Committee Sub-\ac{DAO} may cancel or delay an action in the Main \ac{DAO} or Sub-\acp{DAO}.
	\item If the other members disagree with the decision, they can call a vote of the Committee and get the decision overturned through a simple majority vote.
	The losing party may choose to ragequit (leave the Sub-\ac{DAO}) before the decision is enacted.
	\item The vote’s result will be invalid if:
	\begin{enumerate}
		\item Not all Committee members vote AND the voting period
		has lasted fewer than 7 days or longer than 30 days.
		For clarity, if all Committee members vote, the vote will be considered valid irrespective of the duration.
		\item Fewer than 50\% of Committee members have voted.
	\end{enumerate}

\end{enumerate}

\end{document}
