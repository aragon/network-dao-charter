%!TEX root = network-dao-charter.tex
% SPDX-License-Identifier: CC-BY-SA-4.0

\charterchapter{Sub-DAO Agreements}
\label{chap:SubDAOAgreements}

\begin{enumerate}
	
	\sectionitem{General Provisions}
	\begin{enumerate}
		
		\subsectionitem{Agreements}
		\begin{enumerate}
			\item The Sub-\ac{DAO} is bound primarily to the Aragon Network Charter, and secondarily to its own rules and/or Agreements.
		\end{enumerate}
		
		\subsectionitem{Membership}
		\begin{enumerate}
			\item The Sub-\ac{DAO} shall initially consist of 3 members.
			\begin{enumerate}
				\item In the first year since the creation of the Sub-\ac{DAO}, 2 of these members will be elected by the Aragon Association and 1 member will be elected through an election proposal in the Main \ac{DAO}.
				\item In year 2, 1 of these members will be elected by the Aragon Association and 2 members will be elected through an election proposal in the Main \ac{DAO}.
				\item In year 3, all members will be elected through an election proposal in the Main \ac{DAO}.
				\item The term of each member shall be of 1 year.
				\item Should multiple Committee members be reelected for more than 3 consecutive terms, the member that has been elected for the longest period of time will be discharged and replaced with a new Committee member.
				The discharged member(s) can be re-elected after 2 years.
				\item Only natural persons can be members of the Sub-\ac{DAO}, verified using a decentralized identity solution, chosen by the Main \ac{DAO}.
			\end{enumerate}
		\end{enumerate}
		
		\subsectionitem{Termination of Membership}
		\begin{enumerate}
			\item If a member commits a serious breach (as determined by the Main \ac{DAO}) of the Charter, the Main \ac{DAO} may at any time remove said member of a Sub-\ac{DAO}.
			The member may appeal the decision using Aragon Court.
			\item All proposals for termination will be carried out by an Aragon Voice vote.
			\item Any appeals shall be carried out using Aragon Court.
		\end{enumerate}
		
		\subsectionitem{Compensation \& Expenses}
		\begin{enumerate}
			\item Each Committee member of a Sub-\ac{DAO} shall be paid a monthly fee of 200~\ac{ANT} per month.
			\item Committee members shall be reimbursed by the Executive \ac{DAO} of any reasonable expenses incurred in the performance of their duties.
		\end{enumerate}
	
	\end{enumerate}
	
	\sectionitem{Executive Sub-DAO}
	\begin{enumerate}
		
		\subsectionitem{Responsibilities}
		\begin{enumerate}
			
			\item In particular the Executive \ac{DAO} has the following powers:
			\begin{enumerate}
				\item Pay members of other Sub-\acp{DAO}
				\item Make grants to other community members at their discretion, providing:
				\item such transactions are disclosed transparently on the Aragon Forum website
				\item An Escrow is used to hold funds until the completion of the agreed-upon deliverable.
				\item The deliverable has been fully assessed by the members of the Executive \ac{DAO}.
				\item Pay suppliers of the Aragon network, providing such transactions are disclosed transparently on the Aragon Forum.
			\end{enumerate}
			
			\item And Executive \ac{DAO} members have the following responsibilities:
			\begin{enumerate}
				\item Keep an up to date record of their activities and use of funds.
				\item Hold a General Meeting (online or offline) every fortnight and keep a record of the meeting available to the \glspl{ANTHolder}.
				\item Not miss any more than 3 consecutive General Meetings without providing a valid excuse (medical or force majeure) or a public explanation to \glspl{ANTHolder}.
			\end{enumerate}

		\end{enumerate}
	
		\subsectionitem{Decision-Making Process} 
		\begin{enumerate}
			
			\item The Executive \ac{DAO} uses a lazy-consensus process as follows:
			\begin{enumerate}
				
				\item Any member of the Executive \ac{DAO} can schedule an action in the \ac{DAO}.
				\item If another member disagrees or wishes to discuss said action, provided that said action has not been backed by a majority vote of the Executive \ac{DAO} members in \gls{AragonVoice}, they can:
				\begin{enumerate}
					\item Speak directly with the member who scheduled the action, preferably through a channel that’s readable asynchronously by the community
					\item and, if needed, cancel the scheduled action to provide enough time for the discussion.
				\end{enumerate}
				
				\item If the discussion between members doesn’t lead to an agreement or can not be scheduled soon enough (as determined subjectively by each member), each member of the executive \ac{DAO} can trigger a vote of the Executive \ac{DAO} members using Aragon Voice to resolve the dispute through a simple majority.
				The vote must be open for a minimum of 4 days and no longer than 14 days.
				\item The vote’s result will be invalid if:
				\begin{enumerate}
					\item Not all Committee members vote AND the voting period has lasted less than 14 days or longer than 30 days.
					For clarity, if all Committee members vote, the vote will be considered valid irrespective of the duration.
					\item Less than 50\% of Committee members have voted.
				\end{enumerate}
			
			\end{enumerate}
		
		\end{enumerate}
	
	\end{enumerate}
	

	\sectionitem{Tech Committee}
	\begin{enumerate}
		
		\subsectionitem{Responsibilities}		
		\begin{enumerate}
			
			\item In particular, the Tech Committee \ac{DAO} has the following responsibilities:
			\begin{enumerate}		
				\item Review technical proposals in the Main \ac{DAO} and Sub-\acp{DAO}s to assess them from a technical risk perspective.
				\item Remove technical proposals in the Main \ac{DAO} and Sub-\acp{DAO}s that represent a material technical risk to the project.
				\item Approve technical proposals they believe would be beneficial to the Aragon project and DO NOT require a 3rd party technical security audit due to being low risk.
				\item Suspend technical proposals they believe would be beneficial to the Aragon project, pending the completion of a 3rd party technical security audit.
				\item When needed, add approved proposals to GitHub and other repositories and merge the code.
				\item Maintaining a list of whitelisted technical security auditors they deem to be sufficiently competent to audit Aragon smart contracts.
			\end{enumerate}
		
		\end{enumerate}
	
		\subsectionitem{Decision-Making Process}
		\begin{enumerate}
			\item The Tech Committee decides on a simple majority basis through an Aragon Voice vote of the Tech Committee members.
			\item The vote’s result will be invalid if:
			\begin{enumerate}
				\item Not all Committee members vote AND the voting period has lasted
				less than 14 days or longer than 30 days.
				For clarity, if all Committee members vote, the vote will be considered valid irrespective of the duration.
				\item Less than 50\% of Committee members have voted.
			\end{enumerate}
		\end{enumerate}
	
	\end{enumerate}
	
	\sectionitem{Compliance Committee Sub-DAO}
	\begin{enumerate}
		
		\subsectionitem{Responsibilities}
		\begin{enumerate}
			
			\item In particular the Compliance Committee \ac{DAO} has the following
			responsibilities:
			\begin{enumerate}
				\item Reviewing all proposals in the AN \ac{DAO} and any-sub \ac{DAO} for
				compliance with this Charter and overall legal compliance,
				providing feedback to proposal creators where appropriate.
				\item Remove any proposals they deem to be non-compliant with any
				part of this Charter or illegal.
			\end{enumerate}
		
		\end{enumerate}

		\subsectionitem{Legal Responsibility}
		\begin{enumerate}
			\item The Compliance Committee members assume full legal responsibility for the approval of any illegal, unlawful, criminal or fraudulent proposal.
		\end{enumerate}
	
		\subsectionitem{Decision-Making Process}
		\begin{enumerate}
			
			\item The Compliance Committee \ac{DAO} operates on an lazy-consensus veto basis as follows:
			\begin{enumerate}
				
				\item Any member of the Compliance Committee Sub-\ac{DAO} may cancel or delay an action in the Main \ac{DAO} or Sub-\acp{DAO}.
				\item If the other members disagree with the decision, they can call a vote of the Committee using Aragon Voice and get the decision overturned through a simple majority vote.
				The losing party may choose to ragequit (leave the Sub-\ac{DAO}) before the decision is enacted.
				\item The vote’s result will be invalid if:
				\begin{enumerate}
					\item Not all Committee members vote AND the voting period
					has lasted less than 7 days or longer than 30 days.
					For clarity, if all Committee members vote, the vote will be considered valid irrespective of the duration.
					\item Less than 50\% of Committee members have voted.
				\end{enumerate}
			
			\end{enumerate}
		
		\end{enumerate}
	
	\end{enumerate}

\end{enumerate}
